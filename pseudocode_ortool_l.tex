% Pseudocódigo sucinto do algoritmo ortool_l.py
\documentclass{article}
\usepackage{algorithm}
\usepackage{algpseudocode}
\begin{document}

\section*{Pseudocódigo do Algoritmo de Posicionamento de Servidores Fog}

\begin{algorithm}[H]
\caption{Posicionamento de Servidores Fog (ORTools)}
\begin{algorithmic}[1]
\State Receber matrizes de latências e capacidades, parâmetros de latência/capacidade máxima/mínima, nó cloud
\State Inicializar solver de programação linear inteira
\State Definir variáveis de decisão $x[i]$ (servidor fog) e $y[i][j]$ (nó $i$ servido por $j$)
\State Adicionar restrições:
    \begin{itemize}
        \item Cada nó deve ser servido por um servidor
        \item Se $i$ é servido por $j$, então $j$ é servidor
        \item Servidor serve a si mesmo
        \item Capacidade e latência entre nós e cloud
        \item Cloud é sempre servidor
    \end{itemize}
\State Definir função objetivo: minimizar número de servidores e latência média
\State Resolver o problema
\State Extrair solução: quais nós são servidores, latência média
\end{algorithmic}
\end{algorithm}

\section*{Complexidade}
O algoritmo resolve um problema de Programação Linear Inteira Mista (MILP) com \(N\) nós, \(O(N^2)\) variáveis e restrições. A complexidade é exponencial no pior caso devido à natureza do MILP, embora o tamanho do modelo cresça apenas quadraticamente com \(N\), o que limita na prática o número de nós que podem ser resolvidos em tempo razoável.


\end{document}
